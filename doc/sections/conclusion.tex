\part{总结与展望}

YaDNS是一个简单的DNS中继服务器,能够转发客户端请求,实现对远端服务器返回资源的缓存,也能够加载本地的域名记录。除此以外,YaDNS实现了DoH,能够作为一个本地的DoH代理,安全化DNS流量。YaDNS采用了成熟的开发框架,使用了高效的数据结构,并且经过了测试。在最后,我从收获、问题、展望三个角度,来对本次课程设计进行总结。

\paragraph{收获} 通过对YaDNS的设计,我获益颇丰。一方面,通过对DNS中继服务器的设计与实现,我更加透彻地了解了DNS协议的约定、机制与原理。在进行开发实现的过程中,我对其的认识与掌握也越发深刻。除此以外,通过对一个应用层网络协议的研究与实现,我对网络编程的基本方法与注意事项有了更为切实的体会与理解。通过在实践中真正地、透彻地了解一个网络协议,我对于计算机网络的认知与掌握更为扎实了。最后,YaDNS让我接触到了较为底层的编程,对C语言这样贴近机器底层的语言,与相应的开发模式与方式都有了真切、全面的体会。对一个DNS服务器的实现,让我对底层编程的技能掌握地更加透彻与熟稔。

\paragraph{问题} DNS协议算得上应用层中较为简单的协议。DNS报文的格式非常直观简洁,而RFC 1035标准将DNS协议讲解的非常简明透彻,通过对它的阅读,事实上在协议的实现过程中,并没有遇到很大的困难。为我带来最大困难的是 \lstinline{libuv} 较为复杂的逻辑与对C这样底层语言编程经验的不足。一方面,由于 \lstinline{libuv} 实现的并发模型较为复杂,库函数底层的实现也繁多而难以阅读,导致在出现与其相关的错误时,调试难度较大,常常没有头绪,需要花较多时间。另一方面,由于对C语言编程经验的不足,导致时常容易出现低级错误。尤其是内存的分配、释放、管理上,稍不留神,就可能出现内存的反复释放或内存泄漏问题。但通过对于上面两个问题的解决,我也在这些方面积累了许多经验,在成功实现了YaDNS的全部功能,攻克这些困难之后,我感到受益匪浅。

\paragraph{展望} 在对YaDNS的设计与实现中,我攻克了很多难关,也发现了一些不足之处。一方面,我对于底层编程的经验太少,虽然YaDNS让我更好地掌握了C语言的使用,加深了对网络协议的理解,但在以后的时间中,我也要看重对自己这方面编程能力的培养。另一方面,YaDNS作为一个简单的中继服务器,也有许多需要改进的地方。例如,由于本地记录往往是固定的,可以对存储本地记录的查找树的每一个节点的子孙节点列表进行排序,在查找时,对子孙使用二分法查找,能够进一步提升效率。除此以外,DoT也是实现安全的DNS流量非常主流的方式,可以增加DoT转发方式来进一步使YaDNS的功能更为健全。

综上所述,在YaDNS课程设计的设计、开发、调试过程中,我遇到了许多困难,也通过解决问题收益匪浅,巩固了知识,加强了技能。回顾过去,展望未来,在接下来的学习中,我应当继续努力,巩固能力,弥补不足,不断前行。
