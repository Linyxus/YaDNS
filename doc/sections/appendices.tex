\part{附录}
%\section{Hash函数性能分析}
%\label{sec:hash-func-benchmark}

\section{为什么要使用 \lstinline{libuv} ?}
\label{sec:why-libuv}

YaDNS直接使用 \lstinline{libuv} 实现的事件循环模型而没有自己通过系统调用来实现并发,主要有下面的一些原因。

\paragraph{开发效率}

\lstinline{libuv} 提供了一整套基于事件循环的并发接口。可以很大地提升开发效率。除此以外,DoH实现所依赖的 \lstinline{libcurl} 也提供的Multi接口也已很容易地和 \lstinline{libuv}提供的接口相结合,只需要比较少的新增代码,就能够将 Curl 句柄结合入 \lstinline{libuv} 提供的并发模型中。

\paragraph{可移植性}

虽然系统提供的系统调用同样可以实现基于循环的并发。但不同系统提供的相似功能的调用是不同的。下表给出了不同系统提供的调用。

\begin{table}[h]
  \centering
  \begin{tabular}{cc}
    \toprule
    操作系统 & 系统调用 \\
    \midrule
    Windows & \texttt{select} \\
    Linux & \texttt{epoll} \\
    macOS (BSD) & \texttt{kqueue} \\
    \bottomrule
  \end{tabular}
  \caption{不同系统提供的系统调用}
\end{table}

因而,若自己调用这些系统调用来实现并发,可移植性将大打折扣,需要将并发循环相关的代码使用不同系统的API进行重写,较为繁荣,且不具备较大意义。\lstinline{libuv}已经为我们做好了对于不同底层API的封装,提供了统一、成熟、较为易用的接口。因而使用它事实上更加合理。

\section{编译与构建}

YaDNS构建依赖于如下工具。
\begin{itemize}
  \item CMake
  \item Ninja (Optional)
  \item Conan
\end{itemize}

在构建时,首先创建构建输出文件夹,并使用Conan安装依赖库:
\begin{verbatim}
mkdir -p build/default
cd build/default
conan install ../..
\end{verbatim}

然后,使用CMake生成Ninja构建文件:
\begin{verbatim}
cmake -GNinja ../..
\end{verbatim}

最后,使用Ninja进行编译:
\begin{verbatim}
ninja
\end{verbatim}

在编译成功后,可以运行单元测试:
\begin{verbatim}
./bin/ctest
\end{verbatim}
而编译好的YaDNS可执行文件也在 \texttt{./build/default/bin/} 文件夹中。

如果不想使用Ninja,也可以使用CMake默认的Makefile系统进行编译:
\begin{verbatim}
cmake ../..
make
\end{verbatim}
